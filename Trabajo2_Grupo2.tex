% Options for packages loaded elsewhere
\PassOptionsToPackage{unicode}{hyperref}
\PassOptionsToPackage{hyphens}{url}
%
\documentclass[
]{article}
\usepackage{amsmath,amssymb}
\usepackage{lmodern}
\usepackage{iftex}
\ifPDFTeX
  \usepackage[T1]{fontenc}
  \usepackage[utf8]{inputenc}
  \usepackage{textcomp} % provide euro and other symbols
\else % if luatex or xetex
  \usepackage{unicode-math}
  \defaultfontfeatures{Scale=MatchLowercase}
  \defaultfontfeatures[\rmfamily]{Ligatures=TeX,Scale=1}
\fi
% Use upquote if available, for straight quotes in verbatim environments
\IfFileExists{upquote.sty}{\usepackage{upquote}}{}
\IfFileExists{microtype.sty}{% use microtype if available
  \usepackage[]{microtype}
  \UseMicrotypeSet[protrusion]{basicmath} % disable protrusion for tt fonts
}{}
\makeatletter
\@ifundefined{KOMAClassName}{% if non-KOMA class
  \IfFileExists{parskip.sty}{%
    \usepackage{parskip}
  }{% else
    \setlength{\parindent}{0pt}
    \setlength{\parskip}{6pt plus 2pt minus 1pt}}
}{% if KOMA class
  \KOMAoptions{parskip=half}}
\makeatother
\usepackage{xcolor}
\usepackage[margin=1in]{geometry}
\usepackage{color}
\usepackage{fancyvrb}
\newcommand{\VerbBar}{|}
\newcommand{\VERB}{\Verb[commandchars=\\\{\}]}
\DefineVerbatimEnvironment{Highlighting}{Verbatim}{commandchars=\\\{\}}
% Add ',fontsize=\small' for more characters per line
\usepackage{framed}
\definecolor{shadecolor}{RGB}{248,248,248}
\newenvironment{Shaded}{\begin{snugshade}}{\end{snugshade}}
\newcommand{\AlertTok}[1]{\textcolor[rgb]{0.94,0.16,0.16}{#1}}
\newcommand{\AnnotationTok}[1]{\textcolor[rgb]{0.56,0.35,0.01}{\textbf{\textit{#1}}}}
\newcommand{\AttributeTok}[1]{\textcolor[rgb]{0.77,0.63,0.00}{#1}}
\newcommand{\BaseNTok}[1]{\textcolor[rgb]{0.00,0.00,0.81}{#1}}
\newcommand{\BuiltInTok}[1]{#1}
\newcommand{\CharTok}[1]{\textcolor[rgb]{0.31,0.60,0.02}{#1}}
\newcommand{\CommentTok}[1]{\textcolor[rgb]{0.56,0.35,0.01}{\textit{#1}}}
\newcommand{\CommentVarTok}[1]{\textcolor[rgb]{0.56,0.35,0.01}{\textbf{\textit{#1}}}}
\newcommand{\ConstantTok}[1]{\textcolor[rgb]{0.00,0.00,0.00}{#1}}
\newcommand{\ControlFlowTok}[1]{\textcolor[rgb]{0.13,0.29,0.53}{\textbf{#1}}}
\newcommand{\DataTypeTok}[1]{\textcolor[rgb]{0.13,0.29,0.53}{#1}}
\newcommand{\DecValTok}[1]{\textcolor[rgb]{0.00,0.00,0.81}{#1}}
\newcommand{\DocumentationTok}[1]{\textcolor[rgb]{0.56,0.35,0.01}{\textbf{\textit{#1}}}}
\newcommand{\ErrorTok}[1]{\textcolor[rgb]{0.64,0.00,0.00}{\textbf{#1}}}
\newcommand{\ExtensionTok}[1]{#1}
\newcommand{\FloatTok}[1]{\textcolor[rgb]{0.00,0.00,0.81}{#1}}
\newcommand{\FunctionTok}[1]{\textcolor[rgb]{0.00,0.00,0.00}{#1}}
\newcommand{\ImportTok}[1]{#1}
\newcommand{\InformationTok}[1]{\textcolor[rgb]{0.56,0.35,0.01}{\textbf{\textit{#1}}}}
\newcommand{\KeywordTok}[1]{\textcolor[rgb]{0.13,0.29,0.53}{\textbf{#1}}}
\newcommand{\NormalTok}[1]{#1}
\newcommand{\OperatorTok}[1]{\textcolor[rgb]{0.81,0.36,0.00}{\textbf{#1}}}
\newcommand{\OtherTok}[1]{\textcolor[rgb]{0.56,0.35,0.01}{#1}}
\newcommand{\PreprocessorTok}[1]{\textcolor[rgb]{0.56,0.35,0.01}{\textit{#1}}}
\newcommand{\RegionMarkerTok}[1]{#1}
\newcommand{\SpecialCharTok}[1]{\textcolor[rgb]{0.00,0.00,0.00}{#1}}
\newcommand{\SpecialStringTok}[1]{\textcolor[rgb]{0.31,0.60,0.02}{#1}}
\newcommand{\StringTok}[1]{\textcolor[rgb]{0.31,0.60,0.02}{#1}}
\newcommand{\VariableTok}[1]{\textcolor[rgb]{0.00,0.00,0.00}{#1}}
\newcommand{\VerbatimStringTok}[1]{\textcolor[rgb]{0.31,0.60,0.02}{#1}}
\newcommand{\WarningTok}[1]{\textcolor[rgb]{0.56,0.35,0.01}{\textbf{\textit{#1}}}}
\usepackage{longtable,booktabs,array}
\usepackage{calc} % for calculating minipage widths
% Correct order of tables after \paragraph or \subparagraph
\usepackage{etoolbox}
\makeatletter
\patchcmd\longtable{\par}{\if@noskipsec\mbox{}\fi\par}{}{}
\makeatother
% Allow footnotes in longtable head/foot
\IfFileExists{footnotehyper.sty}{\usepackage{footnotehyper}}{\usepackage{footnote}}
\makesavenoteenv{longtable}
\usepackage{graphicx}
\makeatletter
\def\maxwidth{\ifdim\Gin@nat@width>\linewidth\linewidth\else\Gin@nat@width\fi}
\def\maxheight{\ifdim\Gin@nat@height>\textheight\textheight\else\Gin@nat@height\fi}
\makeatother
% Scale images if necessary, so that they will not overflow the page
% margins by default, and it is still possible to overwrite the defaults
% using explicit options in \includegraphics[width, height, ...]{}
\setkeys{Gin}{width=\maxwidth,height=\maxheight,keepaspectratio}
% Set default figure placement to htbp
\makeatletter
\def\fps@figure{htbp}
\makeatother
\setlength{\emergencystretch}{3em} % prevent overfull lines
\providecommand{\tightlist}{%
  \setlength{\itemsep}{0pt}\setlength{\parskip}{0pt}}
\setcounter{secnumdepth}{-\maxdimen} % remove section numbering
\ifLuaTeX
  \usepackage{selnolig}  % disable illegal ligatures
\fi
\IfFileExists{bookmark.sty}{\usepackage{bookmark}}{\usepackage{hyperref}}
\IfFileExists{xurl.sty}{\usepackage{xurl}}{} % add URL line breaks if available
\urlstyle{same} % disable monospaced font for URLs
\hypersetup{
  pdftitle={Trabajo 2: Ténicas de aprendizaje supervisado.},
  pdfauthor={Luís Filipe Milhomem da Silva Paixão y Marcos Folguera Rivera},
  hidelinks,
  pdfcreator={LaTeX via pandoc}}

\title{Trabajo 2: Ténicas de aprendizaje supervisado.}
\author{Luís Filipe Milhomem da Silva Paixão y Marcos Folguera Rivera}
\date{2023-04-24}

\begin{document}
\maketitle

{
\setcounter{tocdepth}{3}
\tableofcontents
}
\hypertarget{el-conjunto-de-datos-oj-del-paquete-islr-contiene-informaciuxf3n-sobre-1070-compras-de-zumo-de-naranja-realizada-por-distintos-clientes-se-anotuxf3-si-el-zumo-de-naranja-era-de-la-marca-citrus-hill-o-minute-maid-ademuxe1s-se-registruxf3-una-serie-de-caracteruxedsticas-17-tanto-del-cliente-como-del-producto.}{%
\section{1. El conjunto de datos OJ del paquete ISLR contiene
información sobre 1070 compras de zumo de naranja realizada por
distintos clientes, se anotó si el zumo de naranja era de la marca
Citrus Hill o Minute Maid, además, se registró una serie de
características (17) tanto del cliente como del
producto.}\label{el-conjunto-de-datos-oj-del-paquete-islr-contiene-informaciuxf3n-sobre-1070-compras-de-zumo-de-naranja-realizada-por-distintos-clientes-se-anotuxf3-si-el-zumo-de-naranja-era-de-la-marca-citrus-hill-o-minute-maid-ademuxe1s-se-registruxf3-una-serie-de-caracteruxedsticas-17-tanto-del-cliente-como-del-producto.}}

\begin{Shaded}
\begin{Highlighting}[]
\FunctionTok{library}\NormalTok{(ISLR)}
\FunctionTok{data}\NormalTok{(OJ)}
\FunctionTok{attach}\NormalTok{(OJ)}
\end{Highlighting}
\end{Shaded}

\begin{longtable}[]{@{}
  >{\raggedright\arraybackslash}p{(\columnwidth - 2\tabcolsep) * \real{0.2500}}
  >{\raggedright\arraybackslash}p{(\columnwidth - 2\tabcolsep) * \real{0.7500}}@{}}
\toprule()
\begin{minipage}[b]{\linewidth}\raggedright
Variable
\end{minipage} & \begin{minipage}[b]{\linewidth}\raggedright
Descripción
\end{minipage} \\
\midrule()
\endhead
Purchase & Marca de zumo comprada por el cliente. Los valores posibles
son ``CH'' (Citrus Hill) y ``MM'' (Minute Maid). \\
WeekofPurchase & Número de semana en que se realizó la compra. Los
valores van del 237 al 282. \\
StoreID & Identificador de la tienda donde se realizó la compra. Los
valores van del 2 al 137. \\
PriceCH & Precio del zumo de naranja de Citrus Hill. \\
PriceMM & Precio del zumo de naranja de Minute Maid. \\
DiscCH & Descuento en el precio del zumo de naranja de Citrus Hill. \\
DiscMM & Descuento en el precio del zumo de naranja de Minute Maid. \\
SpecialCH & Indica si había una promoción especial en el zumo de naranja
de Citrus Hill en la semana de la compra. Los valores posibles son 0 (no
había promoción especial) y 1 (había promoción especial). \\
SpecialMM & Indica si había una promoción especial en el zumo de naranja
de Minute Maid en la semana de la compra. Los valores posibles son 0 (no
había promoción especial) y 1 (había promoción especial). \\
LoyalCH & Tasa de lealtad del cliente hacia la marca Citrus Hill. \\
SalePriceMM & Precio del zumo de naranja de Minute Maid después del
descuento. \\
SalePriceCH & Precio del zumo de naranja de Citrus Hill después del
descuento. \\
PriceDiff & Diferencia en el precio de los dos tipos de zumo de
naranja. \\
Store7 & Indica si la compra se realizó en la tienda número 7. Los
valores posibles son 0 (no se realizó la compra en la tienda número 7) y
1 (se realizó la compra en la tienda número 7). \\
PctDiscMM & Porcentaje de descuento en el precio del zumo de naranja de
Minute Maid. \\
PctDiscCH & Porcentaje de descuento en el precio del zumo de naranja de
Citrus Hill. \\
ListPriceDiff & Diferencia en el precio de lista de los dos tipos de
zumo de naranja. \\
\bottomrule()
\end{longtable}

\hypertarget{a-crea-un-conjunto-de-entrenamiento-con-800-observaciones-y-reserva-el-resto-como-conjunto-de-validaciuxf3n.-recuerda-establecer-una-semilla-para-que-tu-trabajo-sea-reproducible.}{%
\subsection{a) Crea un conjunto de entrenamiento con 800 observaciones y
reserva el resto como conjunto de validación. Recuerda establecer una
semilla para que tu trabajo sea
reproducible.}\label{a-crea-un-conjunto-de-entrenamiento-con-800-observaciones-y-reserva-el-resto-como-conjunto-de-validaciuxf3n.-recuerda-establecer-una-semilla-para-que-tu-trabajo-sea-reproducible.}}

\begin{Shaded}
\begin{Highlighting}[]
\FunctionTok{set.seed}\NormalTok{(}\DecValTok{1}\NormalTok{)}
\NormalTok{train\_index }\OtherTok{\textless{}{-}} \FunctionTok{sample}\NormalTok{(}\DecValTok{1}\SpecialCharTok{:}\FunctionTok{nrow}\NormalTok{(OJ), }\DecValTok{800}\NormalTok{, }\AttributeTok{replace =} \ConstantTok{FALSE}\NormalTok{)}

\NormalTok{train }\OtherTok{\textless{}{-}}\NormalTok{ OJ[train\_index,]}
\NormalTok{valid }\OtherTok{\textless{}{-}}\NormalTok{ OJ[}\SpecialCharTok{{-}}\NormalTok{train\_index,]}
\end{Highlighting}
\end{Shaded}

\hypertarget{b-construye-un-uxe1rbol-de-clasificaciuxf3n-que-permita-predecir-la-marca-de-zumo-que-compraruxe1-cada-cliente-del-conjunto-de-entrenamiento.-haz-una-representaciuxf3n-y-utiliza-la-funciuxf3n-summary-para-obtener-el-error-de-clasificaciuxf3n-sobre-el-conjunto-de-entrenamiento.-cuuxe1ntos-nodos-finales-tiene}{%
\subsection{b) Construye un árbol de clasificación que permita predecir
la marca de zumo que comprará cada cliente del conjunto de
entrenamiento. Haz una representación y utiliza la función summary()
para obtener el error de clasificación sobre el conjunto de
entrenamiento. ¿Cuántos nodos finales
tiene?}\label{b-construye-un-uxe1rbol-de-clasificaciuxf3n-que-permita-predecir-la-marca-de-zumo-que-compraruxe1-cada-cliente-del-conjunto-de-entrenamiento.-haz-una-representaciuxf3n-y-utiliza-la-funciuxf3n-summary-para-obtener-el-error-de-clasificaciuxf3n-sobre-el-conjunto-de-entrenamiento.-cuuxe1ntos-nodos-finales-tiene}}

\begin{Shaded}
\begin{Highlighting}[]
\FunctionTok{library}\NormalTok{(tree)}
\FunctionTok{set.seed}\NormalTok{(}\DecValTok{1}\NormalTok{)}
\NormalTok{tree\_oj }\OtherTok{\textless{}{-}} \FunctionTok{tree}\NormalTok{(Purchase }\SpecialCharTok{\textasciitilde{}}\NormalTok{ ., }\AttributeTok{data =}\NormalTok{ train)}
\end{Highlighting}
\end{Shaded}

\hypertarget{representaciuxf3n-gruxe1fica-del-uxe1rbol-de-clasificaciuxf3n}{%
\subsubsection{Representación gráfica del árbol de
clasificación}\label{representaciuxf3n-gruxe1fica-del-uxe1rbol-de-clasificaciuxf3n}}

Representamos al árbol de clásficación gráficamente. Al observar esta
representación gráfica de la toma de decisiones acerca de la marca
\texttt{MM} que sería la marca \emph{Minute Maid} y \texttt{CH} que es
\emph{Citrus Hill}.

\begin{Shaded}
\begin{Highlighting}[]
\FunctionTok{plot}\NormalTok{(tree\_oj)}
\FunctionTok{text}\NormalTok{(tree\_oj)}
\end{Highlighting}
\end{Shaded}

\includegraphics{Trabajo2_Grupo2_files/figure-latex/unnamed-chunk-4-1.pdf}

\hypertarget{error-de-clasificaciuxf3n-sobre-el-conjunto-de-entrenamiento}{%
\subsubsection{Error de clasificación sobre el conjunto de
entrenamiento}\label{error-de-clasificaciuxf3n-sobre-el-conjunto-de-entrenamiento}}

\begin{Shaded}
\begin{Highlighting}[]
\FunctionTok{summary}\NormalTok{(tree\_oj)}
\end{Highlighting}
\end{Shaded}

\begin{verbatim}
## 
## Classification tree:
## tree(formula = Purchase ~ ., data = train)
## Variables actually used in tree construction:
## [1] "LoyalCH"       "PriceDiff"     "SpecialCH"     "ListPriceDiff"
## [5] "PctDiscMM"    
## Number of terminal nodes:  9 
## Residual mean deviance:  0.7432 = 587.8 / 791 
## Misclassification error rate: 0.1588 = 127 / 800
\end{verbatim}

Al aplicar \texttt{summary()} sobre el árbol de clasificación anterior
se puede observar las variables que están en la construcción del mismo.
Tenemos el número de nodos terminales, la desviación residual media y la
tasa de error de clásificación. Este arbol comete un error de 15.88\%.
Ha clasificado mal un total de 127 de 800.

\hypertarget{cuuxe1ntos-nodos-finales-tiene}{%
\subsubsection{¿Cuántos nodos finales
tiene?}\label{cuuxe1ntos-nodos-finales-tiene}}

El árbol de clasificación tiene un total de 9 nodo finales/terminales.

\hypertarget{c-utiliza-el-uxe1rbol-para-predecir-los-datos-de-prueba.-obtuxe9n-las-principales-muxe9tricas-de-evaluaciuxf3n-para-el-conjunto-de-prueba.}{%
\subsection{c) Utiliza el árbol para predecir los datos de prueba. Obtén
las principales métricas de evaluación para el conjunto de
prueba.}\label{c-utiliza-el-uxe1rbol-para-predecir-los-datos-de-prueba.-obtuxe9n-las-principales-muxe9tricas-de-evaluaciuxf3n-para-el-conjunto-de-prueba.}}

Al tener en cuenta el error del árbol, podemos estudiar como se comporta
con el conjunto de pruebas, para realizar predicciones.

\begin{Shaded}
\begin{Highlighting}[]
\NormalTok{tree\_oj\_pred }\OtherTok{\textless{}{-}} \FunctionTok{predict}\NormalTok{(tree\_oj, }\AttributeTok{newdata=}\NormalTok{valid, }\AttributeTok{type=}\StringTok{"class"}\NormalTok{)}
\end{Highlighting}
\end{Shaded}

\begin{Shaded}
\begin{Highlighting}[]
\FunctionTok{mean}\NormalTok{(tree\_oj\_pred }\SpecialCharTok{!=}\NormalTok{ valid}\SpecialCharTok{$}\NormalTok{Purchase)}
\end{Highlighting}
\end{Shaded}

\begin{verbatim}
## [1] 0.1703704
\end{verbatim}

\begin{Shaded}
\begin{Highlighting}[]
\NormalTok{metricas}\OtherTok{\textless{}{-}}\ControlFlowTok{function}\NormalTok{(mc)\{ }\CommentTok{\#mc es una matriz de confusión}
\NormalTok{esp}\OtherTok{\textless{}{-}}\NormalTok{mc[}\DecValTok{1}\NormalTok{,}\DecValTok{1}\NormalTok{]}\SpecialCharTok{/}\FunctionTok{sum}\NormalTok{(mc[,}\DecValTok{1}\NormalTok{])}
\NormalTok{sen}\OtherTok{\textless{}{-}}\NormalTok{mc[}\DecValTok{2}\NormalTok{,}\DecValTok{2}\NormalTok{]}\SpecialCharTok{/}\FunctionTok{sum}\NormalTok{(mc[,}\DecValTok{2}\NormalTok{])}
\NormalTok{ag}\OtherTok{\textless{}{-}}\FunctionTok{sum}\NormalTok{(}\FunctionTok{diag}\NormalTok{(mc))}\SpecialCharTok{/}\FunctionTok{sum}\NormalTok{(mc)}
\NormalTok{eg}\OtherTok{\textless{}{-}}\DecValTok{1}\SpecialCharTok{{-}}\NormalTok{ag}
\NormalTok{m}\OtherTok{\textless{}{-}}\FunctionTok{cbind}\NormalTok{(esp,sen,ag,eg)}
\FunctionTok{colnames}\NormalTok{(m)}\OtherTok{=}\FunctionTok{c}\NormalTok{(}\StringTok{"Especificidad"}\NormalTok{,}\StringTok{"Sensibilidad"}\NormalTok{,}\StringTok{"Exactitud"}\NormalTok{,}\StringTok{"Error global"}\NormalTok{)}
\FunctionTok{rownames}\NormalTok{(m)}\OtherTok{=}\StringTok{"valor"}
\NormalTok{m}
\NormalTok{\}}
\NormalTok{mc\_tree\_oj}\OtherTok{\textless{}{-}}\FunctionTok{table}\NormalTok{(}\AttributeTok{predichos=}\NormalTok{tree\_oj\_pred,}\AttributeTok{verdaderos=}\NormalTok{valid}\SpecialCharTok{$}\NormalTok{Purchase)}
\FunctionTok{metricas}\NormalTok{(mc\_tree\_oj)}
\end{Highlighting}
\end{Shaded}

\begin{verbatim}
##       Especificidad Sensibilidad Exactitud Error global
## valor      0.952381     0.627451 0.8296296    0.1703704
\end{verbatim}

\hypertarget{d-utiliza-la-funciuxf3n-cv.tree-para-podar-el-uxe1rbol.-cuuxe1ntos-nodos-finales-tiene-el-uxe1rbol-podado-construye-el-uxe1rbol-podado-calcula-el-error-de-clasificaciuxf3n-de-este-nuevo-uxe1rbol-sobre-el-conjunto-de-entrenamiento-y-sobre-el-conjunto-de-prueba.-compara-los-resultados-con-los-del-uxe1rbol-sin-podar.}{%
\subsection{d) Utiliza la función cv.tree() para podar el árbol.
¿Cuántos nodos finales tiene el árbol podado? Construye el árbol podado,
calcula el error de clasificación de este nuevo árbol sobre el conjunto
de entrenamiento y sobre el conjunto de prueba. Compara los resultados
con los del árbol sin
podar.}\label{d-utiliza-la-funciuxf3n-cv.tree-para-podar-el-uxe1rbol.-cuuxe1ntos-nodos-finales-tiene-el-uxe1rbol-podado-construye-el-uxe1rbol-podado-calcula-el-error-de-clasificaciuxf3n-de-este-nuevo-uxe1rbol-sobre-el-conjunto-de-entrenamiento-y-sobre-el-conjunto-de-prueba.-compara-los-resultados-con-los-del-uxe1rbol-sin-podar.}}

Utilizamos la función \texttt{cv.tree()} que encuentra el número de
clasificaciones erróneas en función del parámetro de coste-complejidad
(k).

\begin{Shaded}
\begin{Highlighting}[]
\FunctionTok{set.seed}\NormalTok{(}\DecValTok{1}\NormalTok{)}
\NormalTok{tree\_oj\_cv }\OtherTok{\textless{}{-}} \FunctionTok{cv.tree}\NormalTok{(tree\_oj,}\AttributeTok{FUN=}\NormalTok{prune.misclass)}
\NormalTok{tree\_oj\_cv}
\end{Highlighting}
\end{Shaded}

\begin{verbatim}
## $size
## [1] 9 8 7 4 2 1
## 
## $dev
## [1] 145 145 146 146 167 315
## 
## $k
## [1]       -Inf   0.000000   3.000000   4.333333  10.500000 151.000000
## 
## $method
## [1] "misclass"
## 
## attr(,"class")
## [1] "prune"         "tree.sequence"
\end{verbatim}

\begin{Shaded}
\begin{Highlighting}[]
\NormalTok{tree\_oj\_podado }\OtherTok{\textless{}{-}} \FunctionTok{prune.misclass}\NormalTok{(tree\_oj,}\AttributeTok{best=}\DecValTok{8}\NormalTok{)}
\FunctionTok{plot}\NormalTok{(tree\_oj\_podado)}
\FunctionTok{text}\NormalTok{(tree\_oj\_podado)}
\end{Highlighting}
\end{Shaded}

\includegraphics{Trabajo2_Grupo2_files/figure-latex/unnamed-chunk-10-1.pdf}

\hypertarget{cuuxe1ntos-nodos-finales-tiene-el-uxe1rbol-podado}{%
\subsubsection{¿Cuántos nodos finales tiene el árbol
podado?}\label{cuuxe1ntos-nodos-finales-tiene-el-uxe1rbol-podado}}

\begin{Shaded}
\begin{Highlighting}[]
\FunctionTok{summary}\NormalTok{(tree\_oj\_podado)}
\end{Highlighting}
\end{Shaded}

\begin{verbatim}
## 
## Classification tree:
## snip.tree(tree = tree_oj, nodes = 4L)
## Variables actually used in tree construction:
## [1] "LoyalCH"       "PriceDiff"     "SpecialCH"     "ListPriceDiff"
## [5] "PctDiscMM"    
## Number of terminal nodes:  8 
## Residual mean deviance:  0.7598 = 601.8 / 792 
## Misclassification error rate: 0.1588 = 127 / 800
\end{verbatim}

Tiene un total de 8 nodos terminales.

\hypertarget{datos-sobre-el-conjunto-de-entranamiento}{%
\subsubsection{Datos sobre el conjunto de
entranamiento}\label{datos-sobre-el-conjunto-de-entranamiento}}

\begin{Shaded}
\begin{Highlighting}[]
\NormalTok{tree\_oj\_podado\_pred\_ent }\OtherTok{\textless{}{-}} \FunctionTok{predict}\NormalTok{(tree\_oj\_podado, }\AttributeTok{newdata=}\NormalTok{train, }\AttributeTok{type=}\StringTok{"class"}\NormalTok{)}
\NormalTok{mc\_podado\_ent }\OtherTok{\textless{}{-}} \FunctionTok{table}\NormalTok{(}\AttributeTok{predichos=}\NormalTok{tree\_oj\_podado\_pred\_ent, }\AttributeTok{verdaderos=}\NormalTok{train}\SpecialCharTok{$}\NormalTok{Purchase)}
\FunctionTok{metricas}\NormalTok{(mc\_podado\_ent)}
\end{Highlighting}
\end{Shaded}

\begin{verbatim}
##       Especificidad Sensibilidad Exactitud Error global
## valor     0.9278351    0.7079365   0.84125      0.15875
\end{verbatim}

\hypertarget{datos-sobre-el-conjunto-de-validaciuxf3n}{%
\subsubsection{Datos sobre el conjunto de
validación}\label{datos-sobre-el-conjunto-de-validaciuxf3n}}

\begin{Shaded}
\begin{Highlighting}[]
\NormalTok{tree\_oj\_podado\_pred\_val }\OtherTok{\textless{}{-}} \FunctionTok{predict}\NormalTok{(tree\_oj\_podado, }\AttributeTok{newdata=}\NormalTok{valid, }\AttributeTok{type=}\StringTok{"class"}\NormalTok{)}
\NormalTok{mc\_podado\_val }\OtherTok{\textless{}{-}} \FunctionTok{table}\NormalTok{(}\AttributeTok{predichos=}\NormalTok{tree\_oj\_podado\_pred\_val, }\AttributeTok{verdaderos=}\NormalTok{valid}\SpecialCharTok{$}\NormalTok{Purchase)}
\FunctionTok{metricas}\NormalTok{(mc\_podado\_val)}
\end{Highlighting}
\end{Shaded}

\begin{verbatim}
##       Especificidad Sensibilidad Exactitud Error global
## valor      0.952381     0.627451 0.8296296    0.1703704
\end{verbatim}

\hypertarget{comparaciuxf3n-de-los-resultados}{%
\subsubsection{Comparación de los
resultados}\label{comparaciuxf3n-de-los-resultados}}

Comparamos el error de clasificación de este nuevo arból sobre el
conjunto de entranamiento, el conjunto de prueba y con los del arbol sin
podar. Calcumalos los errores con el objetivo de compararlos:

\begin{Shaded}
\begin{Highlighting}[]
\CommentTok{\# Sin podar, sobre el conjunto de validación}
\NormalTok{mc\_tree\_oj\_val }\OtherTok{\textless{}{-}}\FunctionTok{table}\NormalTok{(}\AttributeTok{predichos=}\NormalTok{tree\_oj\_pred,}\AttributeTok{verdaderos=}\NormalTok{valid}\SpecialCharTok{$}\NormalTok{Purchase)}
\NormalTok{tree\_oj\_val\_error }\OtherTok{\textless{}{-}} \FunctionTok{metricas}\NormalTok{(mc\_tree\_oj\_val)[}\DecValTok{4}\NormalTok{]}
\CommentTok{\#Sin podar, sobre el conjunto de entranamiento}
\NormalTok{mc\_tree\_oj\_ent\_error }\OtherTok{\textless{}{-}}\FunctionTok{table}\NormalTok{(}\AttributeTok{predichos=}\FunctionTok{predict}\NormalTok{(tree\_oj, }\AttributeTok{newdata=}\NormalTok{train, }\AttributeTok{type=}\StringTok{"class"}\NormalTok{),}\AttributeTok{verdaderos=}\NormalTok{train}\SpecialCharTok{$}\NormalTok{Purchase)}
\NormalTok{tree\_oj\_ent\_error }\OtherTok{\textless{}{-}} \FunctionTok{metricas}\NormalTok{(mc\_tree\_oj\_ent\_error)[}\DecValTok{4}\NormalTok{]}

\CommentTok{\#Podados, sobre el conjunto de validación y entrenamiento, respectivamente.}
\NormalTok{mc\_podado\_val\_error }\OtherTok{\textless{}{-}} \FunctionTok{metricas}\NormalTok{(mc\_podado\_val)[}\DecValTok{4}\NormalTok{]}
\NormalTok{mc\_podado\_ent\_error }\OtherTok{\textless{}{-}} \FunctionTok{metricas}\NormalTok{(mc\_podado\_ent)[}\DecValTok{4}\NormalTok{]}

\NormalTok{tabla\_error }\OtherTok{\textless{}{-}} \FunctionTok{data.frame}\NormalTok{(}
  \StringTok{\textquotesingle{}Tipo de error\textquotesingle{}} \OtherTok{=} \FunctionTok{c}\NormalTok{(}\StringTok{\textquotesingle{}Error de clasificación en validación\textquotesingle{}}\NormalTok{,}
                      \StringTok{\textquotesingle{}Error de clasificación en entrenamiento\textquotesingle{}}\NormalTok{,}
                      \StringTok{\textquotesingle{}Error de clasificación en validación (podado)\textquotesingle{}}\NormalTok{,}
                      \StringTok{\textquotesingle{}Error de clasificación en entrenamiento (podado)\textquotesingle{}}\NormalTok{),}
  \StringTok{\textquotesingle{}Valor\textquotesingle{}} \OtherTok{=} \FunctionTok{c}\NormalTok{(tree\_oj\_val\_error, tree\_oj\_ent\_error, mc\_podado\_val\_error, mc\_podado\_ent\_error)}
\NormalTok{)}

\NormalTok{tabla\_error}
\end{Highlighting}
\end{Shaded}

\begin{verbatim}
##                                      Tipo.de.error     Valor
## 1             Error de clasificación en validación 0.1703704
## 2          Error de clasificación en entrenamiento 0.1587500
## 3    Error de clasificación en validación (podado) 0.1703704
## 4 Error de clasificación en entrenamiento (podado) 0.1587500
\end{verbatim}

Comparando los resultados del árbol sin podar sobre el conjunto de
entranamiento presenta un error de \texttt{0.15875} y el de validación
\texttt{0.1703704}. En este caso, el error árbol de clasificación podado
sobre el conjunto de entranamiento presenta un error de
\texttt{0.1587500}, mientras que el error sobre el conjunto de
validación es de \texttt{0.1703704}. Comparando los resultados sin podar
y con poda, tenemos errores iguales uno para el otro. Así que en este
caso, el elegimos el árbol con menor complejidad que es el árbol podado.

\hypertarget{a-la-vista-de-la-proporciuxf3n-de-ventas-de-cada-clase-de-zumo-cuxf3mo-valoras-el-comportamiento-del-uxe1rbol-de-clasificaciuxf3n-intenta-mejorar-este-comportamiento-utilizando-bosques-aleatorios.-quuxe9-tasa-de-error-oob-obtienes-cuuxe1les-son-las-variables-muxe1s-importantes-a-la-hora-de-determinar-quuxe9-zumo-va-a-comprar-el-consumidor}{%
\subsection{A la vista de la proporción de ventas de cada clase de zumo,
¿cómo valoras el comportamiento del árbol de clasificación? Intenta
mejorar este comportamiento utilizando bosques aleatorios. ¿Qué tasa de
error OOB obtienes? ¿Cuáles son las variables más importantes a la hora
de determinar qué zumo va a comprar el
consumidor?}\label{a-la-vista-de-la-proporciuxf3n-de-ventas-de-cada-clase-de-zumo-cuxf3mo-valoras-el-comportamiento-del-uxe1rbol-de-clasificaciuxf3n-intenta-mejorar-este-comportamiento-utilizando-bosques-aleatorios.-quuxe9-tasa-de-error-oob-obtienes-cuuxe1les-son-las-variables-muxe1s-importantes-a-la-hora-de-determinar-quuxe9-zumo-va-a-comprar-el-consumidor}}

Al observar como se comporta este árbol de clasficación podemos estudiar
primeramente que la proporción de ventas de cada marca de zumo para cada
conjunto es:

\begin{Shaded}
\begin{Highlighting}[]
\FunctionTok{prop.table}\NormalTok{(}\FunctionTok{table}\NormalTok{(OJ[train\_index,]}\SpecialCharTok{$}\NormalTok{Purchase)) }\CommentTok{\#Entranamiento}
\end{Highlighting}
\end{Shaded}

\begin{verbatim}
## 
##      CH      MM 
## 0.60625 0.39375
\end{verbatim}

\begin{Shaded}
\begin{Highlighting}[]
\FunctionTok{prop.table}\NormalTok{(}\FunctionTok{table}\NormalTok{(OJ[}\SpecialCharTok{{-}}\NormalTok{train\_index,]}\SpecialCharTok{$}\NormalTok{Purchase)) }\CommentTok{\#Prueba}
\end{Highlighting}
\end{Shaded}

\begin{verbatim}
## 
##        CH        MM 
## 0.6222222 0.3777778
\end{verbatim}

Podemos ver que no hay mucha diferencia entre las proporciones de venta
de cada marca de jugo en los dos conjuntos. Esto sugiere que los
conjuntos de entrenamiento y prueba se dividieron razonablemente bien al
azar del conjunto de datos completo, y el modelo pudo clasificar con
éxito ambas marcas de jugo en ambos conjuntos.

Podemos \textbf{valorar} que el árbol tiene una buena especificidad, lo
que se traduce en la capacidad de identificar correctamente la mayoría
de los casos negativos. Como resultado, hay espacio para mejorar la
capacidad del modelo para identificar con precisión los casos positivos.
Sin embargo, la sensibilidad del modelo es bastante baja, lo que indica
que hay un margen de mejora para identificar correctamente los casos
positivos. La precisión del árbol es buena, pero no excelente. El árbol
tiene margen de mejora para clasificar correctamente algunos casos, como
lo muestra el valor de error global relativamente alto.

\begin{Shaded}
\begin{Highlighting}[]
\FunctionTok{metricas}\NormalTok{(mc\_podado\_val)}
\end{Highlighting}
\end{Shaded}

\begin{verbatim}
##       Especificidad Sensibilidad Exactitud Error global
## valor      0.952381     0.627451 0.8296296    0.1703704
\end{verbatim}

\emph{Intentamos mejorarlo con utilizando bosques aleatorios.}

\hypertarget{bagging}{%
\subsection{Bagging}\label{bagging}}

El objetivo principal de Bagging (Bootstrap Aggregating) es entrenar
múltiples modelos utilizando varias muestras de entrenamiento que se
obtienen al muestrear y reemplazar el conjunto de datos original.

\begin{Shaded}
\begin{Highlighting}[]
\FunctionTok{set.seed}\NormalTok{(}\DecValTok{1}\NormalTok{)}
\FunctionTok{library}\NormalTok{(randomForest)}
\end{Highlighting}
\end{Shaded}

\begin{verbatim}
## randomForest 4.7-1.1
\end{verbatim}

\begin{verbatim}
## Type rfNews() to see new features/changes/bug fixes.
\end{verbatim}

\begin{Shaded}
\begin{Highlighting}[]
\NormalTok{oj\_bag }\OtherTok{\textless{}{-}} \FunctionTok{randomForest}\NormalTok{(Purchase }\SpecialCharTok{\textasciitilde{}}\NormalTok{ ., }\AttributeTok{data=}\NormalTok{OJ, }\AttributeTok{subset=}\NormalTok{train\_index, }\AttributeTok{ntree=}\DecValTok{500}\NormalTok{, }\AttributeTok{mtry=}\DecValTok{18}\NormalTok{, }\AttributeTok{importance =} \ConstantTok{TRUE}\NormalTok{)}
\NormalTok{oj\_bag}
\end{Highlighting}
\end{Shaded}

\begin{verbatim}
## 
## Call:
##  randomForest(formula = Purchase ~ ., data = OJ, ntree = 500,      mtry = 18, importance = TRUE, subset = train_index) 
##                Type of random forest: classification
##                      Number of trees: 500
## No. of variables tried at each split: 17
## 
##         OOB estimate of  error rate: 20.38%
## Confusion matrix:
##     CH  MM class.error
## CH 399  86   0.1773196
## MM  77 238   0.2444444
\end{verbatim}

Obtenemos una estimación \textbf{out-of-bag} (OOB) de 20.38\%.

Al ver la matriz de confusión se puede ver que comete menos error cuando
clasifica como CH, siendo así más fiable para decir cuando se elige CH.

Pintamos la salida anterior, así obtenemos el cómo se comporta el error
a medida que aumentamos el número de árboles.

\begin{Shaded}
\begin{Highlighting}[]
\FunctionTok{plot}\NormalTok{(oj\_bag)}
\end{Highlighting}
\end{Shaded}

\includegraphics{Trabajo2_Grupo2_files/figure-latex/unnamed-chunk-18-1.pdf}

Se observa un gran descenso del error hasta un poco después de los
árboles 110 aproximadamente, cuando ultrapasa este punto se nota un
pequeño aumento del error acompañados de puntos de constancia del error.
Tenemos el error global (línea negra), el error de clasificar como CH
(línea roja) y el error de clasifica como MM (línea negra).

Tenemos un error de clasificación mayor que el del árbol podado. Antes
teníamos con el árbol podado un error de 0.1703704.

\begin{Shaded}
\begin{Highlighting}[]
\NormalTok{mc\_bag }\OtherTok{\textless{}{-}}\NormalTok{ oj\_bag}\SpecialCharTok{$}\NormalTok{confusion[,}\DecValTok{1}\SpecialCharTok{:}\DecValTok{2}\NormalTok{]}
\FunctionTok{metricas}\NormalTok{(mc\_bag)}
\end{Highlighting}
\end{Shaded}

\begin{verbatim}
##       Especificidad Sensibilidad Exactitud Error global
## valor     0.8382353    0.7345679   0.79625      0.20375
\end{verbatim}

Aún así podemos verificar que métricas obtenemos sobre el conjunto de
validación:

\begin{Shaded}
\begin{Highlighting}[]
\NormalTok{oj\_bag\_pred }\OtherTok{\textless{}{-}} \FunctionTok{predict}\NormalTok{(oj\_bag,}\AttributeTok{newdata=}\NormalTok{OJ[}\SpecialCharTok{{-}}\NormalTok{train\_index,])}
\FunctionTok{mean}\NormalTok{(oj\_bag\_pred }\SpecialCharTok{!=}\NormalTok{ OJ[}\SpecialCharTok{{-}}\NormalTok{train\_index,]}\SpecialCharTok{$}\NormalTok{Purchase)}
\end{Highlighting}
\end{Shaded}

\begin{verbatim}
## [1] 0.1851852
\end{verbatim}

Obtenemos una matriz de confusión para este árbol.

\begin{Shaded}
\begin{Highlighting}[]
\FunctionTok{table}\NormalTok{(oj\_bag\_pred,OJ[}\SpecialCharTok{{-}}\NormalTok{train\_index,]}\SpecialCharTok{$}\NormalTok{Purchase)}
\end{Highlighting}
\end{Shaded}

\begin{verbatim}
##            
## oj_bag_pred  CH  MM
##          CH 152  34
##          MM  16  68
\end{verbatim}

\begin{Shaded}
\begin{Highlighting}[]
\NormalTok{mc\_bag\_pred }\OtherTok{\textless{}{-}} \FunctionTok{table}\NormalTok{(oj\_bag\_pred, OJ[}\SpecialCharTok{{-}}\NormalTok{train\_index,]}\SpecialCharTok{$}\NormalTok{Purchase)}
\FunctionTok{metricas}\NormalTok{(mc\_bag\_pred)}
\end{Highlighting}
\end{Shaded}

\begin{verbatim}
##       Especificidad Sensibilidad Exactitud Error global
## valor     0.9047619    0.6666667 0.8148148    0.1851852
\end{verbatim}

Se confirma que no hubo una mejora en el error en ninguno de los dos
conjuntos.

\hypertarget{bosques-aleatorios}{%
\subsection{Bosques aleatorios}\label{bosques-aleatorios}}

Utilizamos la recomendación de utilizar solo sqrt(p) variables
predictoras.

\begin{Shaded}
\begin{Highlighting}[]
\FunctionTok{set.seed}\NormalTok{(}\DecValTok{1}\NormalTok{)}

\NormalTok{popt }\OtherTok{\textless{}{-}} \FunctionTok{trunc}\NormalTok{(}\FunctionTok{sqrt}\NormalTok{(}\DecValTok{18}\NormalTok{))}
\NormalTok{oj\_rf }\OtherTok{\textless{}{-}} \FunctionTok{randomForest}\NormalTok{(Purchase }\SpecialCharTok{\textasciitilde{}}\NormalTok{ ., }\AttributeTok{data=}\NormalTok{OJ, }\AttributeTok{subset=}\NormalTok{train\_index, }\AttributeTok{ntree=}\DecValTok{500}\NormalTok{, }\AttributeTok{mtry=}\NormalTok{popt, }\AttributeTok{importance=}\NormalTok{T)}
\NormalTok{oj\_rf}
\end{Highlighting}
\end{Shaded}

\begin{verbatim}
## 
## Call:
##  randomForest(formula = Purchase ~ ., data = OJ, ntree = 500,      mtry = popt, importance = T, subset = train_index) 
##                Type of random forest: classification
##                      Number of trees: 500
## No. of variables tried at each split: 4
## 
##         OOB estimate of  error rate: 20.5%
## Confusion matrix:
##     CH  MM class.error
## CH 404  81   0.1670103
## MM  83 232   0.2634921
\end{verbatim}

La estimación out-of-bag (OOB) se ha aumentado un poco, al 20.5\%. Aún
así podemos comprobar que métricas podemos tener con el conjunto de
prueba:

\begin{Shaded}
\begin{Highlighting}[]
\NormalTok{oj\_rf\_pred }\OtherTok{\textless{}{-}} \FunctionTok{predict}\NormalTok{(oj\_rf, }\AttributeTok{newdata=}\NormalTok{OJ[}\SpecialCharTok{{-}}\NormalTok{train\_index,])}
\FunctionTok{mean}\NormalTok{(oj\_rf\_pred }\SpecialCharTok{!=}\NormalTok{ OJ[}\SpecialCharTok{{-}}\NormalTok{train\_index,]}\SpecialCharTok{$}\NormalTok{Purchase)}
\end{Highlighting}
\end{Shaded}

\begin{verbatim}
## [1] 0.1703704
\end{verbatim}

El error de clasificación se pone igual que al del árbol podado. Podemos
ver las \textbf{variables más importantes}:

\begin{Shaded}
\begin{Highlighting}[]
\FunctionTok{importance}\NormalTok{(oj\_rf)}
\end{Highlighting}
\end{Shaded}

\begin{verbatim}
##                        CH        MM MeanDecreaseAccuracy MeanDecreaseGini
## WeekofPurchase  6.6382087  8.694264            12.196544        28.946406
## StoreID         7.6603763 14.469461            16.460260        21.499067
## PriceCH         5.1612787  3.421899             7.155901         5.639893
## PriceMM         2.4256239  5.663620             7.349724         6.272887
## DiscCH          0.6629651  6.428414             4.941466         4.262464
## DiscMM          5.0492630  7.617073            11.144651         5.355865
## SpecialCH       2.0614202  6.570047             6.769480         3.445623
## SpecialMM      -2.9449008 11.237982             9.178979         4.321306
## LoyalCH        60.6871553 83.734449            90.777745       155.678184
## SalePriceMM     5.9383574  9.888114            13.154893        11.362361
## SalePriceCH     4.8396405  3.428337             6.358559         7.106994
## PriceDiff       9.1433248 16.052283            18.983649        18.571671
## Store7          6.6628377  9.954334            10.342098         6.045269
## PctDiscMM       6.7526680  6.255180            11.395150         5.868678
## PctDiscCH       1.6837633  5.967948             4.957308         4.298745
## ListPriceDiff   7.7419833  8.323442            11.684408        10.781037
## STORE           5.3836671 11.559578            11.845265        12.273028
\end{verbatim}

\begin{Shaded}
\begin{Highlighting}[]
\NormalTok{x }\OtherTok{\textless{}{-}} \FunctionTok{importance}\NormalTok{(oj\_rf)[,}\DecValTok{4}\NormalTok{] }\SpecialCharTok{/} \FunctionTok{max}\NormalTok{(}\FunctionTok{importance}\NormalTok{(oj\_rf)[,}\DecValTok{4}\NormalTok{])}
\FunctionTok{sort}\NormalTok{(x)}
\end{Highlighting}
\end{Shaded}

\begin{verbatim}
##      SpecialCH         DiscCH      PctDiscCH      SpecialMM         DiscMM 
##     0.02213298     0.02737997     0.02761302     0.02775794     0.03440344 
##        PriceCH      PctDiscMM         Store7        PriceMM    SalePriceCH 
##     0.03622790     0.03769750     0.03883183     0.04029394     0.04565183 
##  ListPriceDiff    SalePriceMM          STORE      PriceDiff        StoreID 
##     0.06925207     0.07298621     0.07883589     0.11929527     0.13809942 
## WeekofPurchase        LoyalCH 
##     0.18593746     1.00000000
\end{verbatim}

Así que, las variables más importantes a la hora de comprar el zumo, son
\texttt{LoyalCH,\ WeekofPurchase,\ StoreID,\ \ PriceDiff,\ STORE,\ SalePriceMM\ y\ ListPriceDiff}
.

\#2. Aplica los métodos de regresión logística, ADL y KNN a los datos
del ejercicio anterior, con la misma muestra de entrenamiento y de test.
Compara las matrices de confusión y los errores medios de clasificación
de los tres métodos y decide con cuál te quedarías

\#En nuestro conjuntode datos OJ hay una varibale cualitativa con la
cual no podemos hacer los metodos de prediccion, por lo cual la quitamos

\begin{Shaded}
\begin{Highlighting}[]
\FunctionTok{library}\NormalTok{(ISLR)}
\FunctionTok{data}\NormalTok{(OJ)}
\FunctionTok{attach}\NormalTok{(OJ)}
\end{Highlighting}
\end{Shaded}

\begin{verbatim}
## The following objects are masked from OJ (pos = 5):
## 
##     DiscCH, DiscMM, ListPriceDiff, LoyalCH, PctDiscCH, PctDiscMM,
##     PriceCH, PriceDiff, PriceMM, Purchase, SalePriceCH, SalePriceMM,
##     SpecialCH, SpecialMM, STORE, Store7, StoreID, WeekofPurchase
\end{verbatim}

\begin{Shaded}
\begin{Highlighting}[]
\NormalTok{OJ\_new }\OtherTok{\textless{}{-}} \FunctionTok{subset}\NormalTok{(OJ, }\AttributeTok{select =} \SpecialCharTok{{-}}\NormalTok{Store7)}

\FunctionTok{set.seed}\NormalTok{(}\DecValTok{1}\NormalTok{)}

\CommentTok{\# Regresión logística}
\NormalTok{modelo.logit }\OtherTok{\textless{}{-}} \FunctionTok{glm}\NormalTok{(Purchase }\SpecialCharTok{\textasciitilde{}}\NormalTok{ ., }\AttributeTok{data=}\NormalTok{OJ\_new,}\AttributeTok{family=}\NormalTok{binomial)}

\FunctionTok{library}\NormalTok{(vcd)}
\end{Highlighting}
\end{Shaded}

\begin{verbatim}
## Loading required package: grid
\end{verbatim}

\begin{verbatim}
## 
## Attaching package: 'vcd'
\end{verbatim}

\begin{verbatim}
## The following object is masked from 'package:ISLR':
## 
##     Hitters
\end{verbatim}

\begin{Shaded}
\begin{Highlighting}[]
\NormalTok{fp }\OtherTok{\textless{}{-}} \ControlFlowTok{function}\NormalTok{(x) \{}
\NormalTok{  variables}\OtherTok{\textless{}{-}}\DecValTok{0}
\NormalTok{  constante}\OtherTok{\textless{}{-}}\FunctionTok{summary}\NormalTok{(modelo.logit)}\SpecialCharTok{$}\NormalTok{coef[,}\StringTok{"Estimate"}\NormalTok{][}\DecValTok{1}\NormalTok{]}
  \ControlFlowTok{for}\NormalTok{(i }\ControlFlowTok{in} \DecValTok{2}\SpecialCharTok{:}\FunctionTok{length}\NormalTok{(}\FunctionTok{summary}\NormalTok{(modelo.logit)}\SpecialCharTok{$}\NormalTok{coef[,}\StringTok{"Estimate"}\NormalTok{])) \{}
\NormalTok{    variables}\OtherTok{\textless{}{-}}\FunctionTok{summary}\NormalTok{(modelo.logit)}\SpecialCharTok{$}\NormalTok{coef[,}\StringTok{"Estimate"}\NormalTok{][i]}\SpecialCharTok{*}\NormalTok{x}
\NormalTok{  \}}
  \FunctionTok{exp}\NormalTok{(constante}\SpecialCharTok{+}\NormalTok{variables)}\SpecialCharTok{/}\NormalTok{(}\DecValTok{1}\SpecialCharTok{+}\FunctionTok{exp}\NormalTok{(constante}\SpecialCharTok{+}\NormalTok{variables))}
\NormalTok{\}}

\CommentTok{\# Hacer predicciones con el modelo}
\NormalTok{predicciones }\OtherTok{\textless{}{-}} \FunctionTok{predict}\NormalTok{(modelo.logit, }\AttributeTok{type=}\StringTok{"response"}\NormalTok{) }\SpecialCharTok{\textgreater{}} \FloatTok{0.5}
\NormalTok{predicciones }\OtherTok{\textless{}{-}} \FunctionTok{ifelse}\NormalTok{(predicciones }\SpecialCharTok{==} \ConstantTok{TRUE}\NormalTok{, }\DecValTok{1}\NormalTok{,}\DecValTok{0}\NormalTok{)}

\CommentTok{\#Matriz de confusion}
\NormalTok{matriz.rl }\OtherTok{\textless{}{-}} \FunctionTok{table}\NormalTok{(OJ\_new}\SpecialCharTok{$}\NormalTok{Purchase, predicciones, }\AttributeTok{dnn=}\FunctionTok{c}\NormalTok{(}\StringTok{"predicciones"}\NormalTok{,}
\StringTok{"observaciones"}\NormalTok{))}
\FunctionTok{metricas}\NormalTok{(matriz.rl)}
\end{Highlighting}
\end{Shaded}

\begin{verbatim}
##       Especificidad Sensibilidad Exactitud Error global
## valor     0.8522895    0.8066158  0.835514     0.164486
\end{verbatim}

\begin{Shaded}
\begin{Highlighting}[]
\CommentTok{\#Podemos graficar el resultado de matriz/tabla de confusión con la función \textasciigrave{}mosaic()\textasciigrave{}:}
\FunctionTok{mosaic}\NormalTok{(matriz.rl,}\AttributeTok{shade=}\NormalTok{T,}\AttributeTok{colorize=}\NormalTok{T,}
\AttributeTok{gp=}\FunctionTok{gpar}\NormalTok{(}\AttributeTok{fill=}\FunctionTok{matrix}\NormalTok{(}\FunctionTok{c}\NormalTok{(}\StringTok{"green"}\NormalTok{,}\StringTok{"red"}\NormalTok{,}\StringTok{"red"}\NormalTok{,}\StringTok{"green"}\NormalTok{),}\DecValTok{2}\NormalTok{,}\DecValTok{2}\NormalTok{)))}
\end{Highlighting}
\end{Shaded}

\includegraphics{Trabajo2_Grupo2_files/figure-latex/unnamed-chunk-27-1.pdf}

\hypertarget{adl}{%
\section{ADL}\label{adl}}

\begin{Shaded}
\begin{Highlighting}[]
\CommentTok{\#Primero comprobamos cuales son las variables mas fuertemente asociadas a Purchase}
\NormalTok{modelo.logit }\OtherTok{\textless{}{-}} \FunctionTok{glm}\NormalTok{(Purchase }\SpecialCharTok{\textasciitilde{}}\NormalTok{ ., }\AttributeTok{data=}\NormalTok{OJ\_new,}\AttributeTok{family=}\NormalTok{binomial)}
\FunctionTok{set.seed}\NormalTok{(}\DecValTok{1}\NormalTok{)}
\NormalTok{coeficientes\_z }\OtherTok{\textless{}{-}} \FunctionTok{summary}\NormalTok{(modelo.logit)}\SpecialCharTok{$}\NormalTok{coef[, }\StringTok{"z value"}\NormalTok{]}
\NormalTok{coeficientes\_z\_ordenados }\OtherTok{\textless{}{-}} \FunctionTok{sort}\NormalTok{(coeficientes\_z, }\AttributeTok{decreasing =} \ConstantTok{TRUE}\NormalTok{)}

\NormalTok{nombres\_variables }\OtherTok{\textless{}{-}} \FunctionTok{row.names}\NormalTok{(}\FunctionTok{summary}\NormalTok{(modelo.logit)}\SpecialCharTok{$}\NormalTok{coef)}
\NormalTok{variables\_fuertemente\_asociadas }\OtherTok{\textless{}{-}}\NormalTok{ nombres\_variables[}\FunctionTok{which}\NormalTok{(coeficientes\_z }\SpecialCharTok{\%in\%}\NormalTok{ coeficientes\_z\_ordenados[}\DecValTok{1}\SpecialCharTok{:}\DecValTok{10}\NormalTok{])]}

\NormalTok{variables\_fuertemente\_asociadas}
\end{Highlighting}
\end{Shaded}

\begin{verbatim}
##  [1] "(Intercept)"    "WeekofPurchase" "StoreID"        "PriceCH"       
##  [5] "DiscCH"         "DiscMM"         "SpecialCH"      "SpecialMM"     
##  [9] "PctDiscCH"      "STORE"
\end{verbatim}

\begin{Shaded}
\begin{Highlighting}[]
\CommentTok{\#Usamos las 9 variables mas significativas para hacer el analisis discriminante lineal}
\FunctionTok{library}\NormalTok{(MASS)}
\FunctionTok{set.seed}\NormalTok{(}\DecValTok{1}\NormalTok{);}
\NormalTok{modelo.qda }\OtherTok{\textless{}{-}} \FunctionTok{qda}\NormalTok{(Purchase }\SpecialCharTok{\textasciitilde{}}\NormalTok{ PriceCH}\SpecialCharTok{+}\NormalTok{PriceMM}\SpecialCharTok{+}\NormalTok{DiscCH}\SpecialCharTok{+}\NormalTok{DiscMM}\SpecialCharTok{+}\NormalTok{SpecialCH}\SpecialCharTok{+}\NormalTok{SpecialMM}\SpecialCharTok{+}\NormalTok{PctDiscMM}\SpecialCharTok{+}\NormalTok{PctDiscCH}\SpecialCharTok{+}\NormalTok{STORE, }\AttributeTok{data=}\NormalTok{OJ\_new, }\AttributeTok{CV=}\ConstantTok{TRUE}\NormalTok{)}
\FunctionTok{head}\NormalTok{(modelo.qda}\SpecialCharTok{$}\NormalTok{class)}
\end{Highlighting}
\end{Shaded}

\begin{verbatim}
## [1] CH MM CH MM MM CH
## Levels: CH MM
\end{verbatim}

\begin{Shaded}
\begin{Highlighting}[]
\NormalTok{prediccion.qda}\OtherTok{\textless{}{-}}\NormalTok{modelo.qda}\SpecialCharTok{$}\NormalTok{class}
\NormalTok{matriz.qda}\OtherTok{\textless{}{-}}\FunctionTok{table}\NormalTok{(}\AttributeTok{Predichos=}\NormalTok{prediccion.qda,}\AttributeTok{Verdaderos=}\NormalTok{Purchase)}
\NormalTok{matriz.qda}
\end{Highlighting}
\end{Shaded}

\begin{verbatim}
##          Verdaderos
## Predichos  CH  MM
##        CH 524 228
##        MM 129 189
\end{verbatim}

\begin{Shaded}
\begin{Highlighting}[]
\NormalTok{m\_qda }\OtherTok{\textless{}{-}} \FunctionTok{metricas}\NormalTok{(matriz.qda)}
\NormalTok{m\_qda}
\end{Highlighting}
\end{Shaded}

\begin{verbatim}
##       Especificidad Sensibilidad Exactitud Error global
## valor     0.8024502    0.4532374 0.6663551    0.3336449
\end{verbatim}

\begin{Shaded}
\begin{Highlighting}[]
\CommentTok{\#Podemos graficar el resultado de matriz/tabla de confusión con la función \textasciigrave{}mosaic()\textasciigrave{}:}
\FunctionTok{mosaic}\NormalTok{(matriz.qda,}\AttributeTok{shade=}\NormalTok{T,}\AttributeTok{colorize=}\NormalTok{T,}\AttributeTok{gp=}\FunctionTok{gpar}\NormalTok{(}\AttributeTok{fill=}\FunctionTok{matrix}\NormalTok{(}\FunctionTok{c}\NormalTok{(}\StringTok{"green"}\NormalTok{,}\StringTok{"red"}\NormalTok{,}\StringTok{"red"}\NormalTok{,}\StringTok{"green"}\NormalTok{),}\DecValTok{2}\NormalTok{,}\DecValTok{2}\NormalTok{)))}
\end{Highlighting}
\end{Shaded}

\includegraphics{Trabajo2_Grupo2_files/figure-latex/unnamed-chunk-28-1.pdf}

\hypertarget{knn}{%
\section{KNN}\label{knn}}

\begin{Shaded}
\begin{Highlighting}[]
\FunctionTok{set.seed}\NormalTok{(}\DecValTok{1}\NormalTok{)}
\NormalTok{indices }\OtherTok{\textless{}{-}} \FunctionTok{sample}\NormalTok{(}\DecValTok{1}\SpecialCharTok{:}\FunctionTok{nrow}\NormalTok{(OJ\_new), }\DecValTok{800}\NormalTok{, }\AttributeTok{replace =} \ConstantTok{FALSE}\NormalTok{)}

\NormalTok{validacion }\OtherTok{\textless{}{-}}\NormalTok{ OJ\_new[indices,]}
\NormalTok{entrenamiento }\OtherTok{\textless{}{-}}\NormalTok{ OJ\_new[}\SpecialCharTok{{-}}\NormalTok{indices,]}
\NormalTok{clasificador}\OtherTok{\textless{}{-}}\NormalTok{Purchase}

\FunctionTok{library}\NormalTok{(class)}
\NormalTok{kmax}\OtherTok{=}\DecValTok{15}
\NormalTok{errork }\OtherTok{\textless{}{-}} \FunctionTok{c}\NormalTok{()}
\ControlFlowTok{for}\NormalTok{(i }\ControlFlowTok{in} \DecValTok{1}\SpecialCharTok{:}\NormalTok{kmax)\{}
\NormalTok{  x}\OtherTok{\textless{}{-}}\FunctionTok{knn}\NormalTok{(}\AttributeTok{train=}\NormalTok{entrenamiento[, }\SpecialCharTok{{-}}\DecValTok{1}\NormalTok{],}\AttributeTok{test=}\NormalTok{validacion[, }\SpecialCharTok{{-}}\DecValTok{1}\NormalTok{],}\AttributeTok{cl=}\NormalTok{clasificador[}\SpecialCharTok{{-}}\NormalTok{indices],}\AttributeTok{k=}\NormalTok{i)}
\NormalTok{errork[i]}\OtherTok{\textless{}{-}}\FunctionTok{metricas}\NormalTok{(}\FunctionTok{table}\NormalTok{(}\AttributeTok{Predichos=}\NormalTok{x,}\AttributeTok{Verdaderos=}\NormalTok{clasificador[indices]))[}\DecValTok{4}\NormalTok{]}

\NormalTok{\}}
\CommentTok{\#Dibujamos los valores para saber cual es el valor de k optimo }
\FunctionTok{plot}\NormalTok{(}\FunctionTok{c}\NormalTok{(}\DecValTok{1}\SpecialCharTok{:}\NormalTok{kmax),errork,}\AttributeTok{type=}\StringTok{"l"}\NormalTok{)}
\end{Highlighting}
\end{Shaded}

\includegraphics{Trabajo2_Grupo2_files/figure-latex/unnamed-chunk-29-1.pdf}

\begin{Shaded}
\begin{Highlighting}[]
\CommentTok{\#Ahora probamos usando solo las 9 variables mas significativas}
\NormalTok{predictors }\OtherTok{\textless{}{-}} \FunctionTok{c}\NormalTok{(}\StringTok{"PriceCH"}\NormalTok{, }\StringTok{"PriceMM"}\NormalTok{, }\StringTok{"DiscCH"}\NormalTok{, }\StringTok{"DiscMM"}\NormalTok{,}\StringTok{"SpecialCH"}\NormalTok{, }\StringTok{"SpecialMM"}\NormalTok{, }\StringTok{"PctDiscMM"}\NormalTok{, }\StringTok{"PctDiscCH"}\NormalTok{,}\StringTok{"STORE"}\NormalTok{)}
\NormalTok{errork\_new }\OtherTok{\textless{}{-}} \FunctionTok{c}\NormalTok{()}
\ControlFlowTok{for}\NormalTok{(i }\ControlFlowTok{in} \DecValTok{1}\SpecialCharTok{:}\NormalTok{kmax)\{}
\NormalTok{  x}\OtherTok{\textless{}{-}}\FunctionTok{knn}\NormalTok{(}\AttributeTok{train=}\NormalTok{entrenamiento[, predictors],}\AttributeTok{test=}\NormalTok{validacion[, predictors],}\AttributeTok{cl=}\NormalTok{clasificador[}\SpecialCharTok{{-}}\NormalTok{indices],}\AttributeTok{k=}\NormalTok{i)}
\NormalTok{errork\_new[i]}\OtherTok{\textless{}{-}}\FunctionTok{metricas}\NormalTok{(}\FunctionTok{table}\NormalTok{(}\AttributeTok{Predichos=}\NormalTok{x,}\AttributeTok{Verdaderos=}\NormalTok{clasificador[indices]))[}\DecValTok{4}\NormalTok{]}
\NormalTok{\}}
\FunctionTok{plot}\NormalTok{(}\FunctionTok{c}\NormalTok{(}\DecValTok{1}\SpecialCharTok{:}\NormalTok{kmax),errork\_new,}\AttributeTok{type=}\StringTok{"l"}\NormalTok{)}
\end{Highlighting}
\end{Shaded}

\includegraphics{Trabajo2_Grupo2_files/figure-latex/unnamed-chunk-29-2.pdf}

\begin{Shaded}
\begin{Highlighting}[]
\CommentTok{\#Mejora significativamente usando las 9 variables , por lo cual utilizaremos estas 9 variables y un k=8}

\NormalTok{  x}\OtherTok{\textless{}{-}}\FunctionTok{knn}\NormalTok{(}\AttributeTok{train=}\NormalTok{entrenamiento[, predictors],}\AttributeTok{test=}\NormalTok{validacion[, predictors],}\AttributeTok{cl=}\NormalTok{clasificador[}\SpecialCharTok{{-}}\NormalTok{indices],}\AttributeTok{k=}\DecValTok{8}\NormalTok{)}
\NormalTok{matrizKNN}\OtherTok{\textless{}{-}}\FunctionTok{table}\NormalTok{(x, validacion}\SpecialCharTok{$}\NormalTok{Purchase)}
\FunctionTok{metricas}\NormalTok{(matrizKNN)}
\end{Highlighting}
\end{Shaded}

\begin{verbatim}
##       Especificidad Sensibilidad Exactitud Error global
## valor     0.7773196    0.5873016    0.7025       0.2975
\end{verbatim}

\#Conclusion ordenamos los errores obtenidos de los distintos metodos de
prediccion

\begin{Shaded}
\begin{Highlighting}[]
\FunctionTok{library}\NormalTok{(dplyr)}
\end{Highlighting}
\end{Shaded}

\begin{verbatim}
## 
## Attaching package: 'dplyr'
\end{verbatim}

\begin{verbatim}
## The following object is masked from 'package:MASS':
## 
##     select
\end{verbatim}

\begin{verbatim}
## The following object is masked from 'package:randomForest':
## 
##     combine
\end{verbatim}

\begin{verbatim}
## The following objects are masked from 'package:stats':
## 
##     filter, lag
\end{verbatim}

\begin{verbatim}
## The following objects are masked from 'package:base':
## 
##     intersect, setdiff, setequal, union
\end{verbatim}

\begin{Shaded}
\begin{Highlighting}[]
\NormalTok{adl\_metricas }\OtherTok{\textless{}{-}} \FunctionTok{metricas}\NormalTok{(matriz.qda)[}\DecValTok{4}\NormalTok{]}
\NormalTok{rl\_metricas }\OtherTok{\textless{}{-}} \FunctionTok{metricas}\NormalTok{(matriz.rl)[}\DecValTok{4}\NormalTok{]}
\NormalTok{knn\_metricas }\OtherTok{\textless{}{-}} \FunctionTok{metricas}\NormalTok{(matrizKNN)[}\DecValTok{4}\NormalTok{]}

\NormalTok{tabla\_metricas }\OtherTok{\textless{}{-}} \FunctionTok{data.frame}\NormalTok{(Métricas }\OtherTok{=} \FunctionTok{c}\NormalTok{(}\StringTok{"Discriminante Lineal"}\NormalTok{, }\StringTok{"Regresión Logística"}\NormalTok{, }\StringTok{"KNN V. Cercanos"}\NormalTok{),}
                             \AttributeTok{Valor\_Error =} \FunctionTok{c}\NormalTok{(adl\_metricas, rl\_metricas, knn\_metricas))}

\NormalTok{tabla\_metricas\_ }\OtherTok{\textless{}{-}}\NormalTok{ tabla\_metricas }\SpecialCharTok{\%\textgreater{}\%} \FunctionTok{arrange}\NormalTok{(}\FunctionTok{desc}\NormalTok{(Valor\_Error))}
\NormalTok{tabla\_metricas\_}
\end{Highlighting}
\end{Shaded}

\begin{verbatim}
##               Métricas Valor_Error
## 1 Discriminante Lineal   0.3336449
## 2      KNN V. Cercanos   0.2975000
## 3  Regresión Logística   0.1644860
\end{verbatim}

\begin{Shaded}
\begin{Highlighting}[]
\CommentTok{\#Segun los datos obtenidos el mejor metodo para realizar la prediccion el de regresión logística disminuyendo el error hasta la mitad del valor de los otros errores obtenidos }
\end{Highlighting}
\end{Shaded}


\end{document}
